\section{Introduction}
Speaker recognition is a field in non-linear signal processing and pattern recognition in which a speaker is identified. The speaker must be already known by the system. This is achieved by having a training set containing voice clips of the speaker. Different algorithms can then classify the speaker using a set of features extracted from the training voice clips.

The idea behind the project is to differentiate between three persons voices using the methods and categorisers learned in the course non-linear signal processing and pattern recognition (TINONS) at the department of engineering at Aarhus university. 
The voices of all authors was recorded on a single computer using a built-in microphone. The audio files are imported in MATLAB and processed before dividing it into a training set and a verification set.
The data sets are used in a multitude of different classification algorithms in order to compare the performance. The comparison is used to discuss the viability of the classification algorithms. Lastly the results and discussion is used in order to draw a conclusion in the project.

%------------------------------------------------